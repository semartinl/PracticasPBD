\chapter{Introducción}\label{cap:intro}

El capítulo \ref {cap:intro} está dedicado a una introducción al tema del trabajo, describiendo las ideas generales del problema en foco y su importancia. Deberán todavía explicitarse los objetivos del trabajo, aclarando la estructura del informe e indicadas las convenciones tipográficas.

Puede funcionar como una propuesta para el \gls{MUI} de la \gls{FEFT} en la \gls{Unex}, una de las \glspl{IES}.

\section{Contexto}

Debe haber un marco introductorio que describa el contexto en el que se inserta el trabajo, haciendo referencia a la propuesta original del proyecto, que debe figurar en el primer apéndice del documento.

La utilización de las imágenes siempre debe ir acompañada de la referencia correspondiente (Figura~\ref{fig:latex}).

\begin{figure}
\centering
\includegraphics[width=0.5\columnwidth]{images/lion_large.png}
\caption{Uso de \LaTeX.}
\label{fig:latex}
\end{figure}

\section{Objetivos}

Los objetivos del trabajo deben ser presentados de forma clara y compatible con la propuesta original del proyecto. En el caso de que los objetivos originales hayan sido reformulados, deben presentarse las razones objetivas que condujeron a esta reformulación.

Idealmente, se debe incluir un cronograma del proyecto, indicando explícitamente las tareas realizadas y el tiempo dedicado a cada una. Existiendo un cronograma en la propuesta original del proyecto, deberán justificarse eventuales discrepancias con el cronograma real.

\section{Estrutura do Documento}

La estructura efectivamente adoptada para el resto del informe se suele aclarar en esta sección, usando texto similar a: `` El resto del informe está organizado de la siguiente manera: en el capítulo 2 se describe ...; en el capítulo 3 ...; ...; finalmente, el último capítulo presenta las conclusiones y direcciones de trabajo futuro.



